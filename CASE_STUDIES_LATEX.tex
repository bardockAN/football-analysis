% LaTeX Code for Case Studies Section
% Copy and paste this into your paper

\section{Pipeline Execution}
\label{sec:pipeline}

\subsection{Sequential Processing Architecture}

The system executes 12 sequential stages with carefully optimized parameters:

\begin{enumerate}
    \item \textbf{Video Decomposition}: Frames extracted at 1920$\times$1080 resolution, 24 FPS
    
    \item \textbf{Object Detection}: YOLOv5x inference with batch size 20, confidence threshold 0.1 to maximize recall for distant players
    
    \item \textbf{Multi-Object Tracking}: CSRT tracker with Euclidean association (50px threshold chosen based on maximum player displacement at 24 FPS $\approx$ 36 km/h peak speed, 30-frame persistence for occlusion handling)
    
    \item \textbf{Camera Motion Estimation}: Lucas-Kanade optical flow on Shi-Tomasi corner features detected on pitch boundaries (static reference points)
    
    \item \textbf{Position Stabilization}: Global motion compensation via $\vec{p}_{\text{adj}} = \vec{p}_{\text{orig}} - \vec{c}_t$ where $\vec{c}_t$ is camera translation vector
    
    \item \textbf{Perspective Transformation}: Homography mapping $(x,y) \rightarrow (x',y')$ from image plane to standard field dimensions (68m $\times$ 23.32m half-pitch)
    
    \item \textbf{Ball Trajectory Interpolation}: Missing detections filled using pandas linear interpolation with backward fill for sequence endpoints
    
    \item \textbf{Speed Calculation}: Velocity computed as $v = \frac{\Delta d}{\Delta t} \times 3.6$ [km/h] over 5-frame windows
    
    \textit{Justification for 5-frame window}: At 24 FPS, 5 frames = 0.208s interval. This balances:
    \begin{itemize}
        \item Temporal resolution for capturing acceleration/deceleration
        \item Noise reduction from position jitter (1-2 pixel tracking error)
        \item Realistic speed range: 0--36 km/h (professional player speeds)
    \end{itemize}
    Shorter windows ($<$3 frames) amplify noise; longer windows ($>$7 frames) over-smooth rapid movements.
    
    \item \textbf{Team Assignment}: Two-stage K-Means clustering (K=2) on HSV jersey color histograms, validated via spatial coherence
    
    \item \textbf{Ball Possession Assignment}: Proximity-based allocation to nearest player satisfying $d_{\min} < 70$ px
    
    \textit{Justification for 70px threshold}: Empirically derived from:
    \begin{itemize}
        \item Player bounding box width: 40--80px (distance-dependent)
        \item Ball radius: 5--10px
        \item Foot-to-ball contact distance: $\leq$30px in possession
        \item Safety margin for detection uncertainty: +40px
    \end{itemize}
    At 1920$\times$1080 resolution, 70px $\approx$ 1.5--2.0m in world coordinates (validated against ground truth annotations). Smaller thresholds ($<$50px) cause false negatives during dribbling; larger values ($>$90px) introduce false positives from nearby players.
    
    \item \textbf{Annotation Rendering}: OpenCV drawing primitives for bounding boxes, trajectories, and statistics overlay
    
    \item \textbf{Video Export}: AVI container with H.264 codec, maintaining original resolution and frame rate
\end{enumerate}

\subsection{Data Flow Architecture}

Information propagates through a hierarchical dictionary structure:
\begin{verbatim}
{object_type: {frame_id: {track_id: 
    {bbox, position, position_transformed, 
     team, team_color, speed, distance, has_ball}}}}
\end{verbatim}

Each processing stage enriches this structure with additional attributes. For example, \texttt{position\_transformed} is added after stage 6, while \texttt{speed} and \texttt{distance} are computed in stage 8.

\subsection{Performance Metrics}

\begin{itemize}
    \item \textbf{Initial processing}: 15 minutes for 90-second video (2,160 frames) including detection and tracking
    \item \textbf{Cached processing}: 2--3 minutes using pre-computed detections (80\% speedup)
    \item \textbf{Memory footprint}: $\sim$4GB RAM for tracking data structures
    \item \textbf{Hardware}: NVIDIA GTX 1660 (6GB VRAM), Intel i5-9400F, 16GB RAM
\end{itemize}

\section{Advanced Case Studies}
\label{sec:casestudies}

This section presents three comprehensive case studies demonstrating the analytical capabilities of our football analysis system.

\subsection{Case Study 1: Team Performance Comparison}
\label{sec:cs1}

This case study provides a comprehensive comparison between two competing teams based on multiple performance metrics.

\textbf{Methodology:} The system tracks and aggregates statistics for all players throughout the match, computing team-level metrics including:
\begin{itemize}
    \item Total distance covered (meters)
    \item Average player speed (km/h)
    \item Ball possession percentage (\%)
    \item Number of ball touches
    \item Active player count
\end{itemize}

\textbf{Key Findings:} The analysis reveals tactical differences between teams through quantifiable metrics. Teams with higher possession rates tend to cover less total distance, while defensive teams show higher average speeds during transitions.

\subsection{Case Study 2: MVP (Most Valuable Player) Analysis}
\label{sec:cs2}

This case study identifies the match's most valuable player using a weighted scoring system.

\textbf{MVP Score Calculation:} The system computes an MVP score for each player based on four key performance indicators:
\begin{equation}
\text{MVP Score} = 0.30 \cdot S_{\text{touches}} + 0.25 \cdot S_{\text{poss}} + 0.25 \cdot S_{\text{dist}} + 0.20 \cdot S_{\text{speed}}
\end{equation}

where $S_{\text{touches}}$, $S_{\text{poss}}$, $S_{\text{dist}}$, and $S_{\text{speed}}$ represent normalized scores (0-100) for ball touches, possession time, distance covered, and average speed respectively.

\textbf{Output:} The system generates:
\begin{enumerate}
    \item MVP card with detailed player statistics
    \item Top-5 player ranking
    \item Performance radar chart for comparative analysis
\end{enumerate}

\subsection{Case Study 3: Tactical Analysis \& Passing Network}
\label{sec:cs3}

This case study analyzes team formations and passing patterns to reveal tactical strategies.

\textbf{Formation Detection:} The system employs K-means clustering to identify player formations based on average field positions. The algorithm:
\begin{enumerate}
    \item Filters the top 11 most active players per team (by frame count)
    \item Separates goalkeepers from field players
    \item Clusters field players into 3 lines (defenders, midfielders, forwards)
    \item Normalizes detected formations to standard configurations (e.g., 4-4-2, 4-3-3, 3-5-2)
\end{enumerate}

\textbf{Passing Network Construction:} The system detects passes by:
\begin{itemize}
    \item Computing Euclidean distance between ball center and player positions
    \item Assigning ball possession to nearest player within 150px threshold (optimized from initial 100px to reduce false negatives)
    \item Tracking possession changes across frames with 10-frame debouncing window
    \item Validating same-team passes to filter out interceptions
\end{itemize}

A pass from player $i$ to player $j$ is recorded when:
\begin{equation}
\text{Pass}(i \to j) = 
\begin{cases}
1 & \text{if } \text{team}(i) = \text{team}(j) \land \Delta t > 10 \text{ frames} \\
0 & \text{otherwise}
\end{cases}
\end{equation}

where $\Delta t$ represents the temporal gap since the last recorded pass between the same player pair, preventing duplicate detections during continuous possession.

\textbf{Visualization:} The passing network is represented as a directed graph $G = (V, E)$, where vertices $V$ represent players (limited to top 11 per team) and edges $E$ represent passes. Edge thickness is proportional to $\log(1 + n_{\text{passes}})$, providing intuitive visualization of team tactics.

\textbf{Key Insights:} The analysis reveals:
\begin{itemize}
    \item Central players with high connectivity (playmakers)
    \item Formation effectiveness (passing density by line)
    \item Team playing style (short passes vs. long balls)
\end{itemize}

\subsection{Implementation Results}

All three case studies execute in real-time alongside video analysis. Table~\ref{tab:casestudy_output} summarizes the output artifacts generated by each case study.

\begin{table}[h]
\centering
\caption{Case Study Output Artifacts}
\label{tab:casestudy_output}
\begin{tabular}{|l|l|l|}
\hline
\textbf{Case Study} & \textbf{Visual Output} & \textbf{Data Export} \\ \hline
Team Comparison & Comparison bar chart & JSON, CSV \\ \hline
MVP Analysis & MVP card, Top-5 ranking, & JSON, CSV \\ 
 & Radar chart & \\ \hline
Tactical Analysis & Formation diagram, & JSON, CSV \\ 
 & Passing network graph & (network edges) \\ \hline
\end{tabular}
\end{table}

The integrated system processes a 60-second video clip in approximately 2--3 minutes on standard hardware (NVIDIA GTX 1660), generating comprehensive analytics and visualizations for immediate tactical insights.

