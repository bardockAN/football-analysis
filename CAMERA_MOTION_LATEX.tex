\subsection{Camera Motion Compensation}
Camera panning and zooming introduce unwanted motion into player trajectories. To address this, we use Lucas-Kanade optical flow \cite{lucas1981iterative,bouguet2001pyramidal} with Shi-Tomasi corner detection \cite{shi1994good} to estimate camera movement between consecutive frames. We extract 100 feature points from static pitch boundaries (image columns 0--20 and 900--1050) using a 15$\times$15 window size with 2 pyramid levels and convergence threshold $\varepsilon=0.03$. The camera displacement vector is computed as $\vec{c}_t = \arg\max_i \|\vec{f}_t^i - \vec{f}_{t-1}^i\|$, where $\vec{f}_t^i$ represents the position of feature $i$ at frame $t$. Player positions are then adjusted by $\vec{p}_{\text{adj}} = \vec{p}_{\text{orig}} - \vec{c}_t$. 

It is important to note that this approach assumes camera motion typically exceeds player motion ($\|\vec{c}_t\| > \|\vec{v}_{\text{player}}\|$), which is valid for broadcast footage where cameras pan to follow the action. In cases where the camera is stationary or players move faster than camera motion, the pitch boundary features show minimal displacement ($\vec{c}_t \approx \vec{0}$), and the compensation effectively becomes a no-op. Results are cached in \texttt{stubs/camera\_movement\_stub.pkl} for efficiency.
